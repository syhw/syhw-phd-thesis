% la ligne ci-dessous est à insérer obligatoirement dans le préambule du document avant \begin{document}

\usepackage[a4paper]{titre/meta-donnees}

        % les lignes en bas sont à insérer obligatoirement après \begin{document}

        %%%%%%%%%%%%%%%%%%%%%%%%%%%%%%%%%%%%%%%%%%%%%%%%%%%%%%
            %%             Commandes Meta-données               %%
            %%   à renseigner par les auteurs pour générer      %%
            %%     la couverture modèle Univ. Grenoble          %%
            %%%%%%%%%%%%%%%%%%%%%%%%%%%%%%%%%%%%%%%%%%%%%%%%%%%%%%
            %%      Fichier encodé au format ISO-8859-16        %%

            %\Sethpageshift{???mm}   %%optionnel : à décommenter si besoin pour ajout d'espace afin de center la couvérture horizontalement (valeur par défaut est -5.5mm)
            %\Setvpageshift{???mm}   %%optionnel : à décommenter si besoin pour ajout d'espace afin de center la couvérture verticalement (valeur par défaut est -15.5mm)


            %\Universite{}    %%optionnel : à décommenter et à renseigenr si vous voulez changer le non d'université
            %\Grade{}         %%optionnel : à décommenter et à renseigenr si vous voulez changer le grade
            \Specialite{Informatique}
        \Arrete{7 août 2006}
        \Auteur{Gabriel Synnaeve}
        \Directeur{Pierre Bessière}
        %\CoDirecteur{}    %%optionnel : à décommenter et à renseigenr si présence d'un Co-directeur de thèse
            \Laboratoire{Laboratoire d'Informatique de Grenoble}
        \EcoleDoctorale{École Doctorale de Mathématiques, Sciences et Technologies de l'Information, Informatique}         
        \Titre{Bayesian Programming Applied to Multi-Player Video Games}
        %\Soustitre{}      %%optionnel : à décommenter et à renseigenr si présence d'un sous-titre de thèse
            \Depot{XXX}       


        % Commande pour création de nouvelles catégories dans le jury:

            %\UGTNewJuryCategory{...NomDeLaCategorie...}{...Definition...}

        % Exemple \UGTNewJuryCategory{UGTFamille}{Membre de la famille} que nous ajoutons dans la commande \Jury ci-dessous sous la forme \UGTFamille{Jean Rousseau}{(...titre_et_affiliation...s'il_y_en_a...)}


        \Jury{
            \UGTPresident{...Civilité, Prénom et Nom...}{...titre et affiliation...}
%            \UGTPresidente{...Civilité, Prénom et Nom...}{...titre et affiliation...}

            \UGTRapporteur{...Civilité, Prénom et Nom...}{...titre et affiliation...}      %% 1er rapporteur
                \UGTRapporteur{...Civilité, Prénom et Nom...}{...titre et affiliation...}      %% second rapporteur

                \UGTExaminateur{...Civilité, Prénom et Nom...}{...titre et affiliation...}     %% 1er examinateur
                \UGTExaminateur{...Civilité, Prénom et Nom...}{...titre et affiliation...}     %% second examinateur
                \UGTExaminatrice{...Civilité, Prénom et Nom...}{...titre et affiliation...}    %% 3ème examinateur

                \UGTDirecteur{...Civilité, Prénom et Nom...}{...titre et affiliation...}       %% Directeur de thèse
%                \UGTCoDirecteur{...Civilité, Prénom et Nom...}{...titre et affiliation...}     %% Co-Directeur de thèse s'il y en a

                \UGTInvite{...Civilité, Prénom et Nom...}{...titre et affiliation...}
            \UGTInvitee{...Civilité, Prénom et Nom...}{...titre et affiliation...}
        }

%        \MakeUGthesePDG    %% très important pour générer la couvérture de thèse

