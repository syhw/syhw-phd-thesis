\chapter{Strategy}
\label{chapter:strategy}
\begin{quotation}
\textit{Strategy without tactics is the slowest route to victory. Tactics without strategy is the noise before defeat.}\\
%%% \begin{flushright}Sun Tzu (The Art of War, 476-221 BC)\end{flushright}
\textit{All men can see these tactics whereby I conquer, but what none can see is the strategy out of which victory is evolved.}\\
\begin{flushright}Sun Tzu (The Art of War, 476-221 BC)\end{flushright}
\end{quotation}

\lettrine{W}e present our solutions to some of the dimensions of strategies. The main idea is to reduce the complexity encoding all possible variations of strategies to a few strong indicators: the \glos{buildtree} (closely related to the \glos{techtree}) and canonical army compositions. We also try and abstract early game strategies to ``openings'' (as in Chess). Build trees estimation was published at the Annual Conference on Artificial Intelligence and Interactive Digital Entertainment (AAAI AIIDE) 2011 in Palo Alto \citep{SYNNAEVE:StratPred} and openings prediction was published at Computational Intelligence in Games (IEEE CIG) 2011 in Seoul \citep{SYNNAEVE:OpeningPred}.

\ifthenelse{\equal{\myebookformat}{false}}{
\chaptertoc
}{}

\newglossaryentry{EM}{name=EM,description={expectation-maximization, an iterative method to optimize parameters of statistical models depending on unobserved variables. The expectation (E) step gives the likelihood depending on the latent variables, and the maximization (M) step computes maximizing parameters for the expectation of this likelihood.}}

\newglossaryentry{BIC}{name=BIC,description={Bayesian information criterion, a score for model selection. For a given model with $n$ data points, $k$ parameters and $L$ the maximum likelihood, $BIC= -2 \ln(L) + k \ln(n)$}}

\newglossaryentry{GMM}{name=GMM,description={Gaussian mixture model, a generative model in which observations are the outcomes of mixtures of normal distribution.}}

\begin{figure}[!ht]
\begin{center}
\includegraphics[width=0.84\columnwidth]{images/starcraft_bbq_concept_STRATEGY.pdf}
\end{center}
\caption{Information-centric view of the architecture of the bot, the part concerning this chapter is in the dotted rectangle}
\label{fig:conceptSTRATEGY}
\end{figure}

\begin{itemize}
\item Problem: take the winning strategy in the absolute (knowing everything: complete observations, players intentions, effects of each possible actions).
\item Problem that we solve: take the winning strategy (in average) knowing what we saw from the opponent and our model of the game. 
\item Type: prediction is problem of \textit{inference} or \textit{plan recognition} from \textit{incomplete informations}; adaptation given what we know is a problem of \textit{planning under constraints}.
\item Complexity: simple StarCraft decision problems are \textsc{np}-hard \citep{GamingComplexity}, basic StarCraft strategy with full information (remember that StarCraft is partially observable) is mappable to the Generalized Geography problem and thus is \textsc{pspace}-hard \citep{Lichtenstein78}, Chess \citep{Fraenkel81} and Go (with Japanese ko rules) are \textsc{exptime}-complete \citep{Robson83}. Our solutions are real-time on a laptop. %%% TODO XXX
%\item Results: better resistance to noise, which is fundamental in real setup RTS gameplay (fog of war). Full Bayesian model down to adaptation actions some day?
%\item Conclusion and perspectives: a way to encode and use gameplay/structural knowledge 
\end{itemize}

\section{What is a Strategy?}
% tech tree, opening, army composition, build tree

%In a RTS, players need to gather resources to build military units and defeat their opponents. To that end, they often have \textit{worker units} (or extraction structures) than can gather resources needed to build \textit{workers}, \textit{buildings}, \textit{military units} and \textit{research upgrades}. Workers are often also builders (as in StarCraft) and are weak in fights compared to military units. Resources may have different uses, for instance in StarCraft: minerals are used for everything, whereas gas is only required for advanced buildings or military units, and technology upgrades. Buildings and research upgrades define technology trees (directed acyclic graphs) and each state of a ``tech tree'' allow for different unit type production abilities and unit spells/abilities. The military can be of different types, any combinations of ranged, casters, contact attack, zone attacks, big, small, slow, fast, invisible, flying... Units can have attacks and defenses that counter each others as in rock-paper-scissors. We consider \textit{strategy} the way to expand the tech tree as well as the army composition, because both require long term planning under uncertainty, and a risk-reward trade-off. 

As it is an abstract concept, what constitutes \textit{strategy} is hard to grasp. The definition that we use for strategy is a combination of aggressiveness and ``where do we put the cursor/pointer/slider between economy, technology and military production?''. It is very much related to where we put our resources:
\begin{itemize}
    \item If we prioritize economy, on a short term basis our army strength (numbers) may suffer, but on the longer term we will be able to produce more, or produce equally and still expand our economy or technology in parallel. Also, keep in mind that resources depletes mineral deposits/gas geysers can be exhausted/drained.
    \item If we prioritize technology, on a short term basis our army strength may suffer, on the longer term it can strive with powerful units or units which are adapted to the enemy's.
    \item If we prioritize military production, on a short term basis our army strength will strive, but we will not be able to adapt our army composition to new unit types nor increase our production throughput.
\end{itemize}
Strategy is a question of balance between these three axis, and of knowing when we can attack and when we should stay in defense. Being aggressive is best when we have a military advantage (in numbers or technology) over the opponent. Advantages are gained by investments in these axis and using them when we max out our returns (comparatively to the opponent's).

The challenges for an AI enabled/empower with strategic reasoning is to be able to assess the strategic situation of the game (both for itself and for the opponent) and to position these two pointers as best as it can.

\section{Related Work}

%This work was encouraged by the reading of Weber and Mateas' Data Mining Approach to Strategy Prediction \citep{weberStrat} and the fact that they provided their dataset, that we used. They tried and evaluated several machine learning algorithms on replays that were labeled with strategies (openings) with rules.

There are precedent works of Bayesian plan recognition \citep{BMPR}, even in games with \citep{BayesianRecog} using dynamic Bayesian networks to recognize a user's plan in a multi-player dungeon adventure. Commercial RTS games most commonly use \glos{FSM} \citep{FSM_AIGameProgWisdom2003} to encode strategies and their transitions.

There are related works in the domains of opponent modeling \citep{HsiehS08,schadd2007opponent,Kabanza2010}. The main methods used to these ends are case-based reasoning (CBR) and planning or plan recognition \citep{LTW,CBR_Planning,OntanonCBR,HTNPlanning,Ramirez}.

\cite{LTW} used CBR to perform dynamic plan retrieval extracted from domain knowledge in Wargus (Warcraft II open source clone). \cite{CBR_Planning} base their real-time case-based planning (CBP) system on a plan dependency graph which is learned from human demonstration in Wargus. In \citep{OntanonCBR,PlanRetrieval}, they use CBR and expert demonstrations on Wargus. %They update their previous CPB approach by using ``situation assessment'' for plan retrieval. 
They improve the speed of CPB by using a decision tree to select relevant features. \cite{HsiehS08} based their work on \citep{LTW} and used StarCraft replays to construct states and building sequences. Strategies are choices of building construction order in their model. 

\cite{schadd2007opponent} describe opponent modeling through hierarchically structured models of the opponent behaviour and they applied their work to the Spring RTS (Total Annihilation open source clone). \cite{HTNPlanning} use hierarchical task networks (\glos{HTN}) to model strategies in a first person shooter with the goal to use HTN planners. \cite{Kabanza2010} improve the probabilistic hostile agent task tracker (PHATT \citep{PHATT}, a simulated \glos{HMM} for plan recognition) by encoding strategies as HTN. 

\cite{weberStrat} presented ``a data mining approach to strategy prediction'' and performed supervised learning (from buildings features) on labeled StarCraft replays. In this chapter, for openings prediction, we worked with the same dataset as they did, using their openings labels and comparing it to our own labeling method. %\cite{Weber2010qf} applied this automated replay annotation approach to CBP on StarCraft.

\cite{HMMstrat_RTS_AIIDE11} presented their work at the same conference that we presented \citep{SYNNAEVE:StratPred}. They used an HMM which states are extracted from (unsupervised) maximum likelihood on the dataset. The HMM parameters are learned from unit counts (both buildings and military units) every 30 seconds and Viterbi inference is used to predict the most likely next states from partial observations. 


\section{Perception and Interaction}

\subsection{Buildings}
\label{sec:buildings}

From last chapter (section~\ref{sec:techtree}), we recall the \textit{\glos{techtree}} is a directed acyclic graph which contains the whole technological (buildings and upgrades) development of a player. Also, each unit and building has a \textit{sight range} that provides the player with a view of the map. Parts of the map not in the sight range of the player's units are under \textit{fog of war} and the player ignores what is and happens there. 

%%% BUILD TREES vs TECH TREES TODO
\glos{buildorder}

\subsection{Openings}
\label{sec:openings}

In RTS games jargon, an \textit{\glos{opening}} denotes the same thing than in Chess: an early game plan for which the player has to make choices. In Chess because one can not move many pieces at once (each turn), in RTS games because during the development phase, one is economically limited and has to choose between economic and military priorities and can only open so many tech paths at once. The \glos{opening} corresponds to the first military (tactical) moves that will be performed and, in StarCraft, it corresponds to the 5 (early rushes) to 15 minutes (advanced technology / late push) timespan. 

Players have to find out what opening their opponents are doing to be able to effectively deal with the strategy (army composition) and tactics (military moves: where and when) thrown at them. For that, players scout each other and reason about the incomplete information they can bring together about army and buildings composition. This chapter presents a probabilistic model able to predict the \textit{\glos{opening}} of the enemy that is used in a StarCraft AI competition entry bot. Instead of hard-coding strategies or even considering plans as a whole, we consider the long term evolution of our \glos{techtree} and the evolution of our army composition separately (but steering and constraining each others), as shown in Fig.~\ref{bbq_dataflow}. With this model, our bot asks ``what units should I produce?'' (assessing the whole situation), being able to revise and adapt its production plans.

Later in the game, as the possibilities of strategies ``diverge'' (as in Chess), there are no longer fixed foundations/terms that we can speak of as for openings. Instead, what is interesting is to know the technologies available to the enemy as well as have a sense of the composition of their army. The players have to adapt to each other technologies and armies compositions either to be able to defend or to attack. Some units are more cost-efficient than others against particular compositions. Some combinations of units play well with each others (for instance biological units with ``medics'' or a good ratio of strong contact units backed with fragile ranged or artillery units). Finally, some units can be game changing by themselves like cloaking units, detectors, massive area of effects units... 

\subsection{Military Units}
\label{sec:militaryunits}


\section{Replays Labeling}

\subsection{Dataset}

We used Weber and Mateas \citep{weberStrat} dataset of labeled replays. It is composed of 8806 StarCraft: Broodwar game logs, the details are given in Table~\ref{tab:datasetweber}. A \glos{matchup} is a set of the two opponents races: Protoss versus Terran (PvT) is a match-up, Protoss versus Zerg (PvZ) is another one. They are distinguished because strategies distribution are very different across match-ups (see Tables~\ref{tab:datasetweber} and \ref{openings_distrib}). Weber and Mateas used logic rules on building sequences to put their labels, concerning only tier 2 strategies (no tier 1 rushes).

\begin{table}[!h]
\begin{center}
\begin{footnotesize}
\begin{tabular}{|l|c|c|c|l|c|c|c|l|c|c|c|}
\hline
opening & PvP & PvT & PvZ & opening & TvP & TvT & TvZ & opening & ZvP & ZvT & ZvZ \\
\hline
FastLegs & 4 & 17 & 108 & Bio & 144 & 41 & 911 & Lurker & 33 & 184 & 1 \\
FastExpand & 119 & 350 & 465 & Unknown & 66 & 33 & 119 & Unknown & 159 & 164 & 212 \\
ReaverDrop & 51 & 135 & 31 & Standard & 226 & 1 & 9 & HydraRush & 48 & 13 & 9 \\
FastObs & 145 & 360 & 9 & SiegeExpand & 255 & 49 & 7 & Standard & 40 & 80 & 1 \\
Unknown & 121 & 87 & 124 & VultureHarass & 134 & 226 & 5 & TwoHatchMuta & 76 & 191 & 738 \\
Carrier & 2 & 8 & 204 & TwoFactory & 218 & 188 & 24 &  HydraMass & 528 & 204 & 14 \\
FastDT & 100 & 182 & 83 & FastDropship & 96 & 90 & 75 & ThreeHatchMuta & 140 & 314 & 35 \\
\hline
total & 542 & 1139 & 1024 & total & 1139 & 628 & 1150 & total & 1024 & 1150 & 1010 \\
\hline
\end{tabular}
\end{footnotesize}
\end{center}
\caption{Weber's dataset with its labels. XvY means the XvY or YvX match-up but the openings numbers are presented for X.}
\label{fig:weberdatasetnumbers}
\end{table}

\subsection{Our Labeling}

\subsubsection{The Pitfall of Logical/Rule-based Labeling}
Openings are closely related to \gloss{buildorder} (BO) but different BO can lead to the same opening and some BO are shared by different openings. Particularly, if we do not take into account the time at which the buildings are constructed, we may have a wrong opening label too often: if an opening consist in having building C as soon as possible, it does not matter if we built B-A-C instead of the standard A-B-C as long as we have built C quickly. That is the kind of case that will be overlooked by logical labeling. For that reason, we tried to label replays with the statistical appearance of key features with a semi-supervised method (see Figure~\ref{replays_labeling}). Indeed, the purpose of our opening prediction model is to help our StarCraft playing bot to deal with rushes and special tactics. This was not the main focus of Weber and Mateas' labels, which follow exactly the build tree. So, we used the key components of openings that we want to be aware of as features for our labeling algorithm as shown in Table~\ref{labels}.

\begin{figure}[h]
\centerline{\includegraphics[width=0.80\columnwidth]{images/replays_labeling.pdf}}
\caption{Data centric view of our semi-supervised labeling of replays}
\label{replays_labeling}
\end{figure}

\subsubsection{Features}

The selection of the features along with the opening labels is the supervised part of our labeling method. The knowledge of the features and openings comes from expert play and the StarCraft Liquipedia\footnote{\url{http://wiki.teamliquid.net/starcraft/}} (a Wikipedia for StarCraft). They are all presented in Table~\ref{labels}. For instance, if we want to find out which replays correspond to the ``fast Dark Templar'' (DT, Protoss invisible unit) opening, we put the time at which the first Dark Templar is constructed as a feature and perform clustering on replays with it. This is what is needed for our playing bot: to be able to know when he has to fear ``fast DT'' opening and build a detector unit quickly to be able to deal with invisibility.

\begin{table}[h] \caption{Opening/Strategies labels of the replays (Weber's and ours are not always corresponding)}
\begin{footnotesize}
\begin{center}
\begin{tabular}{|c|c|ccc|}
\hline
Race
& Weber's labels
& Our labels
& Features 
& Note (what we fear) \\ \hline
Protoss & FastLegs & speedzeal & Legs, GroundWeapons+1 & quick speed+upgrade attack\\
 & FastDT & fast\_dt & DarkTemplar & invisible units\\
 & FastObs & nony & Goon, Range & quick long ranged attack\\
 & ReaverDrop & reaver\_drop & Reaver, Shuttle & tactical attack zone damages \\
 & Carrier & corsair & Corsair & flying units \\
 & FastExpand & templar & Storm, Templar & powerful zone attack \\
 &  & two\_gates & 2ndGateway, Gateway, Zealot & aggressive rush \\
 & Unknown & unknown & (no clear label) & \\ \hline
Terran  & Bio & bio & 3rdBarracks, 2ndBarracks, Barracks & aggressive rush\\ 
 & TwoFactory & two\_facto & 2ndFactory & strong push (long range) \\ 
 & VultureHarass & vultures & Mines, Vulture & aggressive harass, invisible\\ 
 & SiegeExpand & fast\_exp & Expansion, Barracks & economical advantage \\ 
 & Standard & & & \\ 
 & FastDropship & drop & DropShip & tactical attack \\ 
 & Unknown & unknown & (no clear label) & \\ \hline
Zerg & TwoHatchMuta & fast\_mutas & Mutalisk, Gas & early air raid \\
 & ThreeHatchMuta & mutas & 3rdHatch, Mutalisk & massive air raid \\
 & HydraRush & hydras & Hydra, HydraSpeed, HydraRange & quick ranged attack\\
 & Standard & (speedlings) & (ZerglingSpeed, Zergling) & (removed, quick attacks/mobility) \\
 & HydraMass & & & \\
 & Lurker & lurkers & Lurker & invisible and zone damages \\
 & Unknown & unknown & (no clear label) & \\ \hline
\end{tabular}
\label{labels}
\end{center}
\end{footnotesize}
\end{table}

\begin{table}[h]
\caption{Openings distributions for Terran in all the match-ups}
\begin{center}
\begin{tabular}{|c|cc|cc|cc|}
\hline
&  & vs Protoss &  & vs Terran &  & vs Zerg \\
Opening
& Nb
& \begin{scriptsize}Percentage\end{scriptsize}
& Nb
& \begin{scriptsize}Percentage\end{scriptsize}
& Nb
& \begin{scriptsize}Percentage\end{scriptsize} \\ \hline
%%%two\_gates & 332 & & 252 & & 304 & \\
%%%fast\_dt & 7 & & 87 & & 20 & \\
%%%templar & 3 & & 17 & & 22 & \\
% TvP 1007
% TvT 576
% TvZ 872
bio 	    & 62 	& 6.2 	& 25 	& 4.4   & 197 	& 22.6 \\
fast\_exp 	& 438 	& 43.5 	& 377 	& 65.4  & 392 	& 44.9 \\
two\_facto 	& 240 	& 23.8  & 127 	& 22.0  & 116 	& 13.3 \\
vultures 	& 122 	& 12.1  & 3 	& 0.6   & 3 	& 0.3  \\
drop 	    & 52 	& 5.2   & 10 	& 1.7   & 121 	& 13.9 \\
unknown 	& 93 	& 9.3   & 34 	& 5.9   & 43 	& 5.0 \\ \hline
\end{tabular}
\label{openings_distrib}
\end{center}
\end{table}

\subsubsection{Clustering}

For the clustering part, we tried k-means and expectation-maximization (\glos{EM}, with Gaussian mixtures) on Gaussian mixtures with shapes and volumes chosen with a Bayesian information criterion (\glos{BIC}). Best BIC models were almost always the most agreeing with expert knowledge (15/17 labels), so we kept this method. We used the R package Mclust \citep{Mclust,Mclust2} to perform full EM clustering. 

\label{GMMEM}
The Gaussian mixture model (\glos{GMM}) is as follows:
\begin{itemize}
    \item Variables:
    \begin{itemize}
        \item $X_{i \in \llbracket 1 \dots n\rrbracket}$ the features for the $i$th replay/game. For instance for the ``rax\_fe'' opening (Barracks and fast expand), we used the features ``time of the first expansion'' and ``time of the first Barracks'', as is shown in Figure~\ref{TvZraxFE}, and each data point is an $X_i$.
        \item $Op_{i \in \llbracket1 \dots n\rrbracket}$ the opening of game $i$. Instead of doing one clustering with $k$ possible openings, we did $k$ clusterings of 1 opening vs the rest. So for us $Op_i \in \{opening, rest\}$.
    \end{itemize}
    \item Decomposition:
$$\PP(X_{1:n}, Op_{1:n}) = \prod_{i=1}^n \PP(X_i|Op_i)\PP(Op_i)$$
    \item Forms, for the $k$ openings we have, with $f$ features:
    \begin{itemize}
        \item $\PP(X_i|Op_i) = \sum_{j \in \{op,\neg op\}} \mathcal{N}(\mu_j, \sigma_j^2)$ 
        \item $\PP(Op_i) = Bernouilli(p_{op})$: $\PP(Op_i=op) = p_{op}, \PP(Op_i=\neg op) = 1-p_{op}$
    \end{itemize}
    \item Identification (EM with maximum likelihood\footnote{Maximum a posteriori (MAP) would maximize the joint.}):\\
    Let $\theta= (\mu_{1:2k},\sigma_{1:2k})$, initialize $\theta$ randomly, and let\\
 $L(\theta;X) = \PP(X|\theta) = \prod_{i=1}^n \sum_{Op}\PP(X|\theta,Op)\PP(Op)$, iterate until convergence:
    \begin{enumerate}
        \item E-step: $Q(\theta|\theta^{(t)})= \mathrm{E}[\log L(\theta;x,Op)] = \mathrm{E}[\log \prod_{i=1}^n \sum_{Op_i}\PP(x_i|Op_i,\theta)\PP(Op_i)]$
        \item M-step: $\theta^{(t+1)}=\argmax_{\theta} Q(\theta|\theta^{(t)})$
    \end{enumerate}

    \item Question:
$$\PP(Op_{1:n} | X_{1:n}) \propto \PP(X_{1:n}|Op_{1:n})\PP(Op_{1:n})$$
\end{itemize}
The $\sigma_j$ matrices can define Gaussian distribution of any combinations between:
\begin{itemize}
    \item volume (of the normal distributions of each mixtures at a given $\sigma$ value): equal or variable,
    \item shape (of the multidimensional normal distributions of each mixtures): equal or variable, 
    \item orientation: spherical (i.e. no orientation), coordinate axes, equal (any orientation although the same for all mixtures components) or variable (any orientation for any component).
\end{itemize}
We chose the combinations with the best (i.e. smallest) \glos{BIC} score. For a given model with $n$ data points, $m$ parameters and $L$ the maximum likelihood, $BIC= -2 \ln(L) + m \ln(n)$. A $\sigma$ with variable volume, variable shape and variable orientation is also called a full covariance matrix.

\begin{figure}[hp]
\centerline{\includegraphics[width=0.56\columnwidth]{images/PvTfastDT.png}}
\caption{Protoss vs Terran distribution of first appearance of Dark Templars (Protoss invisible unit).}
\label{PvTfastDT}
\end{figure}

\begin{figure}[hp]
\centerline{\includegraphics[width=0.7\columnwidth]{images/TvZraxFE.png}}
\caption{Terran vs Zerg Barracks and first Expansion timings (Terran). The bottom left cluster (squares) is the one labeled as \textit{fast\_exp}.}
\label{TvZraxFE}
\end{figure}

\begin{figure}[hp]
\centerline{\includegraphics[width=0.7\columnwidth]{images/PvPspeedzeal.png}}
\caption{Protoss vs Protoss Ground Attack +1 and Zealot Legs upgrades timings. The bottom left cluster (squares) is the one labeled as \textit{speedzeal}.}
\label{PvPspeedzeal}
\end{figure}

\begin{figure}[hp]
\centerline{\includegraphics[width=0.7\columnwidth]{images/ZvPmutas.png}}
\caption{Zerg vs Protoss time of the third Hatch and first appearance of Mutalisks. The bottom left cluster (squares) is the one labeled as \textit{mutas}.}
\label{ZvPmutas}
\end{figure}

\subsubsection{Labeling, Score Filtering}

We produce ``2 bins clustering'' for each set of features (corresponding to each opening), and label the replays belonging to the cluster with the lower norm of features' appearances (that is exactly the purpose of our features). Figures~\ref{PvPspeedzeal}, \ref{TvZraxFE}
and \ref{ZvPmutas} show the clusters out of EM with the features of the corresponding openings. We thought of clustering because there are two cases in which you build a specific military unit of research a specific upgrade: either it is part of your opening, or it is part of your longer term game plan or even in reaction to the opponent. So the distribution over the time at which a feature appears is bimodal, with one (sharp) mode corresponding to ``opening with it'' and the other for the rest of the games, as can be seen in Figure~\ref{PvTfastDT}.

Finally, some replays are labeled two or three times with different labels (due to the different time of effect of different openings), so we apply a score filtering to transform multiple label replays into unique label ones (see Figure~\ref{replays_labeling}). For that we choose the openings labels that were happening the earliest (as they are a closer threat to the bot in a game setup) if and only if they were also the most probable or at 10\% of probability of the most probable label (to exclude transition boundaries of clusters) for this replay. We find the earliest by comparing the norms of the clusters means in competition. All replays without a label or with multiple labels (\textit{i.e.} which did not had a unique solution in filtering) after the filtering were labeled as \textit{unknown}. We then used this labeled dataset as well as Weber and Mateas' labels in the testing of our Bayesian model for opening prediction.


\section{Build Tree Prediction}
\label{sec:strategyprediction}
The work described in this section can be classified as probabilistic plan recognition. Strictly speaking, we present model-based machine learning used for prediction of plans, while our model is not limited to prediction. We performed two levels of plan recognition, both are learned from the replays: tech tree prediction (unsupervised, this section) and opening prediction (semi-supervised or supervised depending on the labeling method, next section).

\begin{quotation}\textit{
What is of supreme importance in war is to attack the enemy's strategy.}\\
Sun Tzu\end{quotation}

\label{sec:techtreepred}
%\subsubsection{Variables}
%\subsubsection{Decomposition}
%\subsubsection{Forms}
%\subsubsection{Identification (learning)}
%\subsubsection{Questions}

\subsection{Results on StarCraft}


\subsubsection{Prediction}

For each match-up, we ran cross-validation testing with 9/10th of the dataset used for learning and the remaining 1/10th of the dataset used for testing. We ran tests finishing at 5, 10 and 15 minutes to capture all kinds of openings (early to late ones). To measure the predictive capability of our model, we used 3 metrics: 
\begin{itemize}
\item the \textit{final} prediction, which is the opening that is predicted at the end of the test, 
\item the \textit{online twice} (OT), which counts the openings that have emerged as most probable twice a test (so that their predominance is not due to noise),
\item the \textit{online once} $> 3$ (OO3), which counts the openings that have emerged as most probable openings after 3 minutes (so that these predictions are based on really meaningful information).
\end{itemize}
After 3 minutes, a Terran player will have or be building his first supply depot, barracks, refinery (gas), and at least factory or expansion. A Zerg player would have his first overlord, zergling pool, extractor (gas) and most of the time his expansion and lair tech. A Protoss player would have his first pylon, gateway, assimilator (gas), cybernectics core, and sometimes his robotics center, or forge and expansion.

\begin{figure}[htp]
\centerline{\includegraphics[width=0.8\columnwidth]{images/TvP_prediction.png}}
\centerline{\includegraphics[width=0.8\columnwidth]{images/TvPx_prediction.png}}
\caption{Evolution of $\PP(Opening)$ with increasing observations in a TvP match-up, with Weber's labeling on top, our labeling on the bottom. The x-axis corresponds to the construction of buildings.}
\label{prediction}
\end{figure}

Table~\ref{results} sums up all the prediction probabilities (scores) of our model in all the match-ups with both labeling of the game logs. Please note that when an opening is mispredicted, the distribution on openings is often not $\PP(badopening)=1, \PP(others)=0$ and that we can extract some value out of these distributions. Also, we observed that $\PP(Opening=unknown)>\PP(others)$ is often a case of misprediction: our bot would use the next prediction in this case. Figure~\ref{prediction} shows the evolution of the distribution $\PP(Opening)$ during a replay for Weber's and our labelings. Figure~\ref{noise} shows the resistance of our model to noise. We randomly removed some observations (buildings, attributes), from 1 to 15, knowing that for Protoss and Terran we use 16 buildings observations and 17 for Zerg. We think that our model copes well with noise because it backtracks unseen observations: for instance if we have only the $core$ observation, it will work with build trees containing $core$ that will passively infer unseen $pylon$ and $gate$. Also, uncertainty is handled natively.



\begin{figure}[htp]
\centerline{\includegraphics[width=0.7\columnwidth]{images/ZvP2.png}}
\centerline{\includegraphics[width=0.7\columnwidth]{images/PvZ2.png}}
\caption{Two extreme evolutions of the 3 probabilities of opening recognition with increasing noise (15 missing attributes/observations/buildings correspond to 93.75\% missing information for Protoss and Terran openings prediction and 88.23\% of missing attributes for Zerg openings prediction). Zerg opening prediction probabilitly on top, Protoss bottom.}
\label{noise}
\end{figure}

\begin{sidewaystable}[ht] \caption{Prediction probabilities for all the match-ups}
\begin{center}
\begin{tabular}{|c|ccc|ccc|ccc|ccc|ccc|ccc|}
\hline
& \multicolumn{9}{|c|}{Weber and Mateas' labels}
& \multicolumn{9}{|c|}{Our labels}
\\  \hline
& \multicolumn{3}{|c|}{5 minutes}
& \multicolumn{3}{|c|}{10 minutes}
& \multicolumn{3}{|c|}{15 minutes}
& \multicolumn{3}{|c|}{5 minutes}
& \multicolumn{3}{|c|}{10 minutes}
& \multicolumn{3}{|c|}{15 minutes}
\\
\begin{scriptsize}match-up\end{scriptsize}
& final
& OT
& OO3
& final
& OT
& OO3
& final
& OT
& OO3
& final
& OT
& OO3
& final
& OT
& OO3
& final
& OT
& OO3 \\ \hline
PvP & 0.65 & 0.53 & 0.59 & 0.69 & 0.69 & 0.71 & 0.65 & 0.67 & 0.73
& 0.78 & 0.74 & 0.68 & 0.83 & 0.83 & 0.83 & 0.85 & 0.83 & 0.83 \\
PvT & 0.75 & 0.64 & 0.71 & 0.78 & 0.86 & 0.83 & 0.81 & 0.88 & 0.84
& 0.62 & 0.69 & 0.69 & 0.62 & 0.73 & 0.72 & 0.6 & 0.79 & 0.76 \\
PvZ & 0.73 & 0.71 & 0.66 & 0.8 & 0.86 & 0.8 & 0.82 & 0.87 & 0.8
& 0.61 & 0.6 & 0.62 & 0.66 & 0.66 & 0.69 & 0.61 & 0.62 & 0.62 \\
TvP & 0.69 & 0.63 & 0.76 & 0.6 & 0.75 & 0.77 & 0.55 & 0.73 & 0.75
& 0.50 & 0.47 & 0.54 & 0.5 & 0.6 & 0.69 & 0.42 & 0.62 & 0.65 \\
TvT & 0.57 & 0.55 & 0.65 & 0.5 & 0.55 & 0.62 & 0.4 & 0.52 & 0.58
& 0.72 & 0.75 & 0.77 & 0.68 & 0.89 & 0.84 & 0.7 & 0.88 & 0.8 \\
TvZ & 0.84 & 0.82 & 0.81 & 0.88 & 0.91 & 0.93 & 0.89 & 0.91 & 0.93
& 0.71 & 0.78 & 0.77 & 0.72 & 0.88 & 0.86 & 0.68 & 0.82 & 0.81 \\
ZvP & 0.63 & 0.59 & 0.64 & 0.87 & 0.82 & 0.89 & 0.85 & 0.83 & 0.87
& 0.39 & 0.56 & 0.52 & 0.35 & 0.6 & 0.57 & 0.41 & 0.61 & 0.62 \\
ZvT & 0.59 & 0.51 & 0.59 & 0.68 & 0.69 & 0.72 & 0.57 & 0.68 & 0.7
& 0.54 & 0.63 & 0.61 & 0.52 & 0.67 & 0.62 & 0.55 & 0.73 & 0.66 \\
ZvZ & 0.69 & 0.64 & 0.67 & 0.73 & 0.74 & 0.77 & 0.7 & 0.73 & 0.73
& 0.83 & 0.85 & 0.85 & 0.81 & 0.89 & 0.94 & 0.81 &  0.88 & 0.94\\ \hline
\begin{scriptsize}overall\end{scriptsize} & 0.68 & 0.62 & 0.68 & 0.73 & 0.76 & 0.78 & 0.69 & 0.76 & 0.77 
& 0.63 & 0.67 & 0.67 & 0.63 & 0.75 & 0.75 & 0.63 & 0.75 & 0.74 \\ \hline
\end{tabular}
\label{results}
\end{center}
\end{sidewaystable}


\subsubsection{Performances}
The first iteration of this model was not making use of the structure imposed by the game in the form of ``possible build trees'' and was at best very slow, at worst intractable without sampling. With the model presented here, the performances are ready for production as shown in Table~\ref{CPU}. The memory footprint is around 3.5Mb on a 64bits machine. Learning computation time is linear in the number of games logs events ($O(N)$ with $N$ observations), which are bounded, so it is linear in the number of game logs. It can be serialized and done only once when the dataset changes. The prediction computation corresponds to the sum in the question (III.B.5) and so its computational complexity is in $O(N\cdot M)$ with $N$ build trees and $M$ possible observations, as $M << N$, we can consider it linear in the number of build trees (values of $BuildTree$).

\begin{table}[h]
\caption{Extremes of computation time values (in seconds, Core 2 Duo 2.8Ghz)}
\begin{center}
\begin{tabular}{|c|cc|cc|}
\hline
Race
& Nb Games
& Learning time
& Inference $\mu$
& Inference $\sigma^2$ \\ \hline
T (max) & 1036 & 0.197844 & 0.0360234 & 0.00892601 \\
T \begin{tiny}(Terran)\end{tiny} & 567 & 0.110019 & 0.030129 & 0.00738386 \\ 
P \begin{tiny}(Protoss)\end{tiny} & 1021 & 0.13513 & 0.0164457 & 0.00370478 \\
P \begin{tiny}(Protoss)\end{tiny} & 542 & 0.056275 & 0.00940027 & 0.00188217 \\ 
Z \begin{tiny}(Zerg)\end{tiny} & 1028 & 0.143851 & 0.0150968 & 0.00334057 \\
Z \begin{tiny}(Zerg)\end{tiny} & 896 & 0.089014 & 0.00796715 & 0.00123551 \\ \hline
\end{tabular}
\label{CPU}
\end{center}
\end{table}



\section{Early Game: Openings}

\subsection{Opening Prediction Model}
We now build up on buildings prediction to make opening prediction % TODO

Our predictive model is a Bayesian program, it can be seen as the ``Bayesian network'' represented in Figure~\ref{BNPrediction}. It is a generative model and this is of great help to deal with the parts of the observations' space where we do not have too much data (RTS games tend to diverge from one another as the number of possible actions grow exponentially). Indeed, we can model our uncertainty by putting a large standard deviation on too rare observations and generative models tend to converge with fewer observations than discriminative ones \citep{Jordan}. Here is the description of our Bayesian program:

\begin{figure}[htp]
\centerline{\includegraphics[width=0.4\columnwidth]{images/OpeningPrediction2.pdf}}
\caption{Graph representation of the opening (and tech tree) prediction model}
\label{BNPrediction}
\end{figure}

\subsubsection{Variables}
\begin{itemize}
\item $BuildTree \in [\emptyset, building_1, building_2, building_1\wedge building_2, techtrees, \dots]$: all the possible building trees for the given race. For instance $\{pylon, gate\}$ and $\{pylon, gate, core\}$ are two different $BuildTrees$.
\item $N\ Observations$: $O_{i \in \llbracket 1\dots N \rrbracket} \in \{0, 1\}$, $O_k$ is $1\ (true)$ if we have seen (observed) the $k$th building (it can have been destroyed, it will stay ``seen'').
\item $Opening$: $Op^t \in [opening_1 \dots opening_M]$ take the various opening values (depending on the race).
\item $LastOpening$: $Op^{t-1} \in [opening_1 \dots opening_M]$, Opening value of the previous time step (allows filtering, taking previous inference into account).
\item $\lambda \in \{0, 1\}$: coherence variable (restraining $BuildTree$ to possible values with regard to $O_\llbracket 1 \dots N \rrbracket$)
\item $Time$: $T \in \llbracket 1\dots P \rrbracket$, time in the game (1 second resolution).
\end{itemize}

At first, we generated all the possible (according to the game rules) $BuildTree$ values (between $\approx 500$ and $1600$ depending on the race). We observed that a lot of possible $BuildTree$ values are too absurd to be performed in a competitive match and were never seen during the learning. So, we restricted $BuildTree$ to have its value in all the build trees encountered in our replays dataset. 
%%%There are 292 build trees for Terran, 176 for Protoss and 113 for Zerg ($\approx 3000$ replays/race), all learned from the (unlabeled) replays.
There are 810 build trees for Terran, 346 for Protoss and 261 for Zerg ($\approx 3000$ replays/race), all learned from the (unlabeled) replays.

\subsubsection{Decomposition}
The joint distribution of our model is the following:
\begin{eqnarray}
    & & \PP(T, BuildTree, O_1 \dots O_N, Op^t, Op^{t-1}, \lambda) \\ 
& = &   \PP(Op^t | Op^{t-1}) \\
    & & \PP(Op^{t-1}) \\
    & & \PP(BuildTree | Op^t) \\
    & & \PP(O_{\llbracket 1 \dots N\rrbracket}) \\
    & & \PP(\lambda | BuildTree, O_{\llbracket 1 \dots N\rrbracket}) \\
    & & \PP(T | BuildTree, Op^t) 
\end{eqnarray}
This can also be see as Figure~\ref{BNPrediction}.

\subsubsection{Forms}
\begin{itemize}
\item $\PP(Op^t | Op^{t-1})$ is optional, we use it as a filter so that the previous inference impacts the current one. We use a functional Dirac:
\begin{eqnarray*}
& & \PP(Op^t|Op^{t-1})\ \  (Dirac)\\
& = & 1\ \mathrm{if}\ Op^t = Op^{t-1}\\
& = & 0\ \mathrm{else}
\end{eqnarray*}
This does not prevent our model to switch predictions, it just uses previous inference posterior $\PP(Op^{t-1}$ to average $\PP(Op^t)$.

\item $\PP(Op^{t-1})$ copied from one inference to another (mutated from $\PP(Op^t)$). The first $\PP(Op^{t-1})$ is bootstrapped with the uniform distribution, we could also use a prior on openings in the given match-up.
\item $\PP(BuildTree | Op^t)$ is learned from the labeled replays. $\PP(BuildTree|Op^t)$ are $\mathrm{card}(\{openings\})$ different histogram over the values of $BuildTree$.
\item $\PP(O_{\llbracket 1 \dots N\rrbracket})$ is unspecified, we put the uniform distribution.
\item $\PP(\lambda | BuildTree, O_{\llbracket 1 \dots N\rrbracket})$ is a functional Dirac that restricts $BuildTree$ values to the ones than can co-exist with the observations.
\begin{eqnarray*}
& & \PP(\lambda = 1 | buildTree, o_{\llbracket 1 \dots N\rrbracket}) \\
& = & 1\ \mathrm{if\ } buildTree \ \mathrm{can\ exist\ with\ } o_{\llbracket 1\dots N\rrbracket} \\
& = & 0\ \mathrm{else}
\end{eqnarray*}
A $BuildTree$ value ($buildTree$) is compatible with the observations if it covers them fully. For instance, $BuildTree=\{pylon, gate, core\}$ is compatible with $o_{\#core} = 1$ but it is not compatible with $o_{\#forge} = 1$. In other words, $buildTree$ is incompatible with $o_{\llbracket 1\dots N\rrbracket}$ \textit{iff} $\{o_{\llbracket 1\dots N\rrbracket} \backslash \{o_{\llbracket 1\dots N\rrbracket} \wedge buildTree\}\} \neq \emptyset$.
\item $\PP(T | BuildTree, Op^t)$ are ``bell shape'' distributions (discretized normal distributions). There is one bell shape per couple $(opening, buildTree)$. The parameters of these discrete Gaussian distributions are learned from the labeled replays.
\end{itemize}

\subsubsection{Identification (learning)}
The learning of the $\PP(BuildTree | Op^t)$ histogram is straight forward counting of occurrences from the labeled replays with ``add-one smoothing'' (Laplace's law of succession \citep{Jaynes}). The learning of the $\PP(T | BuildTree, Op^t)$ bell shapes parameters takes into account the uncertainty of the couples $(buildTree, opening)$ for which we have few observations. Indeed, the normal distribution $\PP(T|buildTree, opening)$ begins with a high $\sigma^2$, and \textbf{not} a Dirac with $\mu$ on the seen $T$ value and $sigma=0$. This accounts for the fact that the first observation may have been an outlier. This learning process is independent on the order of the stream of examples, seeing point A and then B or B and then A in the learning phase produces the same result. 

\begin{figure}[htp]
\centerline{\includegraphics[width=0.7\columnwidth]{images/BW_HeatMap_knowing_ReaverDrop.png}}
\caption{$\PP(Time, BuildTree | Opening^t=ReaverDrop)$}
\label{noise}
\end{figure}


\subsubsection{Questions}
%%%\begin{eqnarray*}
%%%\PP(Op|T=t, O_{1:N}=o_{1:N}, \lambda = 1) \\
%%%\propto \frac{1} {\PP(o_{1:N}).\PP(\lambda).\PP(t)} \\
%%%\sum_{BuildTree} \PP(Op).\PP(BuildTree|Op).\PP(o_{1:N})\\
%%%.\PP(\lambda|BuildTree,o_{1:N}).\PP(t|BuildTree,Op)
%%%\end{eqnarray*}
The question that we will ask in all the benchmarks is:
\begin{eqnarray}
 & \PP(Op|T=t, O_{\llbracket 1\dots N\rrbracket}=o_{\llbracket 1\dots N\rrbracket}, \lambda = 1) \\
 & \propto \PP(Op).\PP(o_{\llbracket 1\dots N\rrbracket})\\
 & \times \sum_{BuildTree} \PP(\lambda | BuildTree, o_{\llbracket 1\dots N\rrbracket})\\
 & .\PP(BuildTree|Op).\PP(t|BuildTree, Op)
\end{eqnarray}
Note that if we see $\PP(BuildTree, Time)$ as a plan, asking $\PP(BuildTree|Opening, Time)$ boils down to use our ``plan recognition'' mode as a planning algorithm, which could provide good approximations of the optimal goal set \citep{Ramirez}. This gives us a distribution on the build trees to follow (build orders) to achieve a given opening.


To sum-up, the full Bayesian program is:
\begin{eqnarray*}
\begin{sideways}\parbox{35mm}{\hspace{-1.3cm}Bayesian program}\end{sideways}
\begin{cases}
\begin{sideways}\parbox{15mm}{\hspace{-0.8cm}Description}\end{sideways}
    \begin{cases}
\begin{sideways}\parbox{35mm}{\hspace{-1.1cm}Specification ($\pi$)}\end{sideways}
        \begin{cases}
        Variables\\
    T, BuildTree, O_1 \dots O_N, Op^t, Op^{t-1}, \lambda \\ 
        Decomposition\\
            \PP(T, BuildTree, O_1 \dots O_N, Op^t, Op^{t-1}, \lambda) \\ 
        =   \PP(Op^t | Op^{t-1}) \PP(Op^{t-1}) \PP(BuildTree | Op^t) \\
            \PP(O_{\llbracket 1 \dots N\rrbracket}) \PP(\lambda | BuildTree, O_{\llbracket 1 \dots N\rrbracket}) \PP(T | BuildTree, Op^t) \\
        Forms\\
            \PP(Op^t | Op^{t-1}) = functional\ Dirac\ \mathrm{(filtering)}\\
            \PP(BuildTree | Op^t) = Categorical(\theta)\\
%%% \mathrm{histogram} \\
            \PP(\lambda | BuildTree, O_{\llbracket 1 \dots N\rrbracket}) = functional\ Dirac\ \mathrm{(coherence)}\\ 
%%%\ of\ observations\ with\ buildtree)}\\
            %%%\PP(T | BuildTree, Op^t) = BellShape(\mu, \sigma)
            \PP(T | BuildTree=bt, Op^t=op) = Discrete\mathcal{N}(\mu_{bt,op},\sigma^2_{bt,op})\\
%%%\mathrm{bell\ shape\ (discrete\ normal)}\\
        \end{cases}\\
    Identification\\
            \PP(BuildTree=bt | Op^t=op) = \theta_{bt,op} = \frac{1 + \mathrm{count}(bt,op)}{\#BuildTree + \mathrm{count}(op)}\\
            (\mu_{bt,op},\sigma_{bt,op})_{\mathrm{ML}} = \argmax_{\mu,\sigma}\PP(T|BuildTree=bt, Op^t=op; \mu,\sigma)\\
%%% \mathrm{maximum\ likelihood\ }\{(\mu,\sigma^2)\}\\
    \end{cases}\\
Question \\
 \PP(Op|T=t, O_{\llbracket 1\dots N\rrbracket}=o_{\llbracket 1\dots N\rrbracket}, \lambda = 1) \\
\end{cases}
\end{eqnarray*}



\subsection{Strengths and Weaknesses}

\begin{table}[ht] 
\begin{footnotesize}
\begin{center}
\begin{tabular}{|l|ccccc|}
\hline
\begin{tiny}Zerg | Protoss\end{tiny} & two gates & fast dt & reaver drop & corsair & nony\\
\hline
speedlings &  0.417& 0.75 & NED & NED & 0.5 \\
lurkers &  NED & 0.493& NED & 0.445 & 0.533\\
fast mutas &  NED & 0.506& 0.5 & 0.526& 0.532\\
\hline
\end{tabular}
\begin{tabular}{|l|cccc|}
\hline
\begin{tiny}Terran | Protoss\end{tiny} & fast dt & reaver drop & corsair & nony\\
\hline
%%% bio &  NED & NED & NED & NED & NED & NED \\
%%% unknown &  NED & NED & 0.5 & NED & NED & NED \\
%%% drop &  NED & NED & NED & NED & NED & NED \\
two facto &  0.552& 0.477& NED & 0.578\\
rax fe &  0.579& 0.478& 0.364& 0.584\\
%%% vultures &  NED & NED & NED & NED & NED & NED \\
\hline
\end{tabular}
\caption{Opening/Strategies labels counted for victories against each others for the PvZ (top, on 1408 games) and PvT (bottom, on 1657 games) match-ups. NED stands for Not Enough Data to conclude a preference/discrepancy towards one opening. The results should be read as win rates of columns openings vs lines openings.}
\end{center}
\end{footnotesize}
\label{table:openingcontingency}
\end{table}


\section{Army Composition}
\subsection{Army Adaptation Model}
We downloaded more than 9000 replays to keep 7708 uncorrupted, 1v1 replays of very high level StarCraft games (pro-gamers leagues and international tournaments) from specialized websites\footnote{\url{http://www.teamliquid.net}, \url{http://www.gosugamers.net}, \url{http://www.iccup.com}}. Then, we ran them using BWAPI\footnote{\url{http://code.google.com/p/bwapi/}} and dumped units positions, pathfinding and regions, resources, orders, vision events, and for attacks (we trigger a robust attack tracking heuristic when a unit dies and there are at least two military units around): types, positions, units engaged on each side, and outcomes. Basically, every BWAPI event was recorded, the dataset, its source code, and this model implementation are freely available\footnote{\url{http://snippyhollow.github.com/bwrepdump/}}. 

%\subsection{Army Composition Model}
In this model, we assume that we have some incomplete knowledge of the opponent's tech tree and a quite complete knowledge of his army composition. We want to know what units we should build now, to adapt our army to their, while staying coherent with our own tech tree and tactics constraints. To that end, we reduce armies to clusters (or mixtures of clusters) that could have generated a given composition. In this lower dimension (usually between 5 to 15 mixture components), %, up to 50 ``hard'' clusters), 
we can reason on which mixture of clusters the opponent is probably going to have, according to his current mixture and his tech tree. As we learned how to pair compositions strengths and weaknesses, we can adapt to this ``future mixture''. This Bayesian program can be seen as the Bayesian network represented in Figure~\ref{BN}. %It is a generative model and this helps dealing with the parts of the observations' space where we do not have too much data (RTS games tend to diverge from one another as the number of possible actions grow exponentially), as generative models tend to converge with fewer observations than discriminative ones \cite{Jordan}. Here is the description of our Bayesian program:

\begin{figure}[htp]
\centerline{\includegraphics[width=0.55\columnwidth]{images/army_composition_model.pdf}}
\caption{Graphical (plate notation) representation of the army composition Bayesian model, gray variables are known while gray hatched variables are given by distributions on their values. $C_t$ can also be known (if a decision was taken on a subsequent model).}
\label{BN}
\end{figure}

\subsubsection{Variables}
\begin{itemize}
    \item $TT^t, ETT^t$ are two $TechTree$ variables, one for us ($TT$) and one for the enemy ($ETT$), at time $t$. $TechTree \in \{\emptyset, \{building_1\}, \{building_2\}, \{building_1\wedge building_2\},$ $\dots\}$: all the possible building trees for the given race (strictly speaking, we here only use build trees). For instance $\{pylon, gate\}$ and $\{pylon, gate, core\}$ are two different $TechTrees$. $TT$ has $V$ possible values and $ETT$ has $V'$ possible values (depending on the faction).
    \item $C_{t,c,m,f}^{t+1}, EC^{t,t+1} \in \{clusters\}$ (see further), our army cluster ($C$), both wanted ($C_{t}$ for $tactics$,$C_{c}$ for $counter$, $C_{m}$ for $merge$) and decided ($C_{f}$ for $final$) at $t+1$, and the enemy's cluster ($EC$) estimated at $t$ from their units that we saw, and estimated at $t+1$ from their (estimated distribution on) tech trees and previous $EC^t$. $C_{t},C_{c},C_{m},C_{f}$ have $K$ values ($K$ units clusters for us) and $EC^t, EC^{t+1}$ have $K'$ values. $C_t$ is a distribution of the clusters we want for tactics (another model) or just a value (for instance we just need a \textit{dropship/shuttle}). $C_c$ is the counter to what we infer the enemy army will be ($EC^{t+1}$). $C_m$ merge our tactics desiderata with our adaptivity, while $C_f$ take the tech tree constraints into account. This can seem tedious but these variables are all to be seen as different states of the two variables $C$ and $EC$, representing army compositions, respectively for us and for the enemy.
    \item $U_{1:N}^{t+1}, EU_{1:M}^t \in [0\dots 1]$, our $N$ unit types ($U$) proportions (indice on type) at time $t+1$, and the $M$ enemy units types ($EU$) at time $t$. For instance, an army with equal numbers of $zealots$ and $dragoons$ (and nothing else) is represented as $\{U_{zealot}=0.5, U_{dragoon}=0.5, \forall ut \neq zealot|dragoon\ U_{ut}=0.0\}$.
    \item $\lambda \in \{0, 1\}$ is a coherence variable unifying $C_{m}^{t+1}$ and $C_{f}^{t+1}$ to possible values with regard to $TT^t$.
\end{itemize}

For tech trees ($TT$ and $ETT$) values, it would be absurd to generate all the possible combinations (for instance no player is going to build four barracks before building the first supply depot), we used all the values that were encountered in our games (replays) dataset. The bases of these trees were buildings, with some types being counted for repetition, for instance a tech tree of $\{pylon, gateway\}$ is not the same as $\{pylon, gateway, gateway\}$: we effectively count the gateway twice. %We added multiple instances of the basic unit producing buildings (gateway, barracks, hatchery), expansions and supply buildings (depot, pylon, ``overlord'' as a building). 
This way, we counted 273 probable tech tree values for Protoss, 211 for Zerg, and 517 for Terran (the ordering of add-on buildings is multiplying tech trees for Terran). Should it happen, we can deal with unseen tech tree either by using the closest one (in set distance) or using an additional value of no knowledge. %See \cite{SYNNAEVE:StratPred} (which deals with build trees, $BT$) for more details on inferring tech trees.

\subsubsection{Decomposition}
The joint distribution of our model is the following:
\begin{eqnarray}
& & \PP(TT^t,C_{t}^{t+1},C_{c}^{t+1},C_{m}^{t+1},C_{f}^{t+1},\\
& & ETT^t,EC^{t},EC^{t+1},U_{1:N}^{t+1},EU_{1:M}^{t}) \\
& = & \PP(EU_{1:M}^t|EC^t)\PP(EC^t|EC^{t+1})\\
& \times & \PP(EC^{t+1}|ETT^t)\PP(ETT^t)\\
& \times & \PP(C_{c}^{t+1}|EC^{t+1})\PP(C_{t}^{t+1})\PP(TT^t)\PP(C_{f}^{t+1}|TT^{t})\\
& \times & \PP(\lambda|C_{f}^{t+1},C_{m}^{t+1})\PP(C_{m}^{t+1}|C_{t}^{t+1},C_{c}^{t+1})\\
& \times & \PP(U_{1:N}^{t+1}|C_{f}^{t+1})
\end{eqnarray}
This can also be see as Figure~\ref{BN}.


%%% TODO GMM

\subsubsection{Forms}
\begin{itemize}
\item Both $\PP(U_{1:N}^{t+1}|C_f^{t+1})$ and $\PP(EU_{1:M}^t|EC^t)$ are results of the clustering of armies compositions ($U_{1:N}$, $EU_{1:M}$) into $EC$ and $C$, and so depend from the clustering model. They are both learned from the data through expectation maximization. In the best model that we tried, they come from a Gaussian mixtures model (GMM) and so are from $\mathcal{N}(\mu_{z_i}, \sigma^2_{z_i})$ with $z_i \sim Categorial(K)$, respectively $Categorical(K')$. %$\PP(U_{1:N}^{t+1}|C_f^{t+1}=c_f) \sim$ 

\item $\PP(EC^t|EC^{t+1})$ is a probability table of dimensions $K' \times K'$ resulting of the temporal dynamic between clusters, that is learned from the dataset with a Laplace succession law (``add one'' smoothing) \cite{Jaynes}.

\item $\PP(ETT^t)$ is a categorical distribution on $V'$ values, i.e. an histogram distribution on the enemy's tech trees. It comes from the strategy prediction model explained in \cite{SYNNAEVE:StratPred}. For us, $TT \sim Categorical(V)$ too, and we know our own tech tree ($TT$) exactly.

\item $\PP(EC^{t+1}|ETT^t)$ is a probability table of dimensions $K' \times V'$ resulting of the inter-dependency between some tech trees and clusters/mixtures. It is learned from the dataset with a Laplace succession law.

\item $\PP(C_c^{t+1}|EC^{t+1})$ is a probability table of dimensions $K \times K'$, which is learned from battles with a Laplace succession law on victories: $\PP(c_c|ec) = \frac{n_{victories}(c_c, ec)+1}{n_{battles}(c_c, ec)+K}$.

\item $\PP(C_t^{t+1})$ is a $Categorical(K)$ (histogram on the $K$ clusters values) coming from a tactical sub-model. Note that it can be degenerate: $\PP(C_t^{t+1}=shuttle)=1.0$, it serves the purpose of merging tactical needs with strategic ones.

\item $\PP(C_f^{t+1}|TT)$ is either a table, as for $P(EC^{t+1}|ETT^t)$, or a functional Dirac distribution which tells which $C_f$ values ($c_f$) are compatible with the current tech tree $TT=tt$. It is compatible if this tech tree ($tt$) allows for building all units present in $c_m$. For instance, $tt=\{pylon, gate, core\}$ is compatible with a $c_m=\{\mu_{U_{zealot}}=0.5, \mu_{U_{dragoon}}=0.5, \forall ut \neq zealot|dragoon\ \mu_{U_{ut}}=0.0\}$, but it is not compatible with $c_m'=\{\mu_{U_{arbiter}}=0.1, \dots\}$.

%\item $\PP(C_m^{t+1}|C_t^{t+1},C_c^{t+1}) = \alpha.\PP(C_t^{t+1})+(1-\alpha).\PP(C_c^{t+1})$ with $\alpha$ the aggressiveness/initiative parameter which can be set fixed, learned, or be variable ($P(\alpha|situation...)$).

\item $\PP(\lambda | C_f^{t+1},C_m^{t+1})$ is a functional Dirac that restricts $C_m$ values to the ones that can co-exist with our tech tree ($C_f$).
\begin{eqnarray*}
& & \PP(\lambda = 1 | C_f, c_m) \\
& = & 1\ \mathrm{if\ } \PP(C_f=c_m) \neq 0\\
& = & 0\ \mathrm{else}
\end{eqnarray*}
This can be (and was) simply implemented as a function. This is not strictly necessary (as one could have $\PP(C_f|TT,C_m)$ which would do the same for $\PP(C_f)=0.0$) but it allows us to have the same table form for $\PP(C_f|TT)$ than for $\PP(EC|ETT)$, should we wish to use the learned co-occurrences tables (with respect to the good race).

\end{itemize}

\subsubsection{Identification (learning)}
The first approach to clustering units was a simple fusion $\PP(U_{1:N},C)\propto \prod_i \PP(U_i|C)$ with rectangular $\PP(U_i|C)$ distributions above thresholds for basic and special (casters) units, and their compositions. So we fixed $\approx 8+4+4*8 = 44$ clusters (for each race) and set their parameters. The problem with this approach is that it requires a high game strategy expertize and understanding of the multiple influences of parameters, while it cannot take into account singular combinations of 3 (or more) unit types. We alternatively learned Gaussian mixture models (GMM) with the expectation-maximization (EM) algorithm on 5 to 15 mixtures with sperical, tied, diagonal and full co-variance matrices \cite{scikit-learn}. We kept the best scoring models according to the Bayesian information criterion \cite{schwarz1978}: $-2 \ln{\max{Likelihood}} + k \ln(n)$, with $k$ the number of parameters and $n$ the number of observations. %used the python library scikit-learn  to learn the Gaussian Mixture Models. 
%We also implemented other clustering methods and variants which are listed in the benchmark.

For the categorical probability tables, we used Laplace rule of succession (``add-one smmoothing'') \cite{Jaynes}: $\PP(A=a|B=b)=\frac{n(a,b)+1}{n(a,b)+n(\neg a,b)+|A|}$.

\subsubsection{Questions}
The question that we ask to know which units to produce is:
\begin{eqnarray}
 & & \PP(U_{1:N}^{t+1} | eu_{1:M}^t, tt^t, \lambda = 1) \\
 & \propto & \sum_{ETT^t, EC^t, EC^{t+1}}[\PP(EC^{t+1}|ETT^t)\PP(ETT^t)\\
 & \times & \PP(eu_{1:M}^t|EC^t) \\
 & \times & \sum_{C_f^{t+1},C_m^{t+1},C_c^{t+1},C_t^{t+1}}[\PP(C_c^{t+1}|EC^{t+1})\\
 & \times & \PP(C_{t}^{t+1})\PP(C_{f}^{t+1}|tt^{t})\\
 & \times & \PP(\lambda|C_{f}^{t+1},C_{m}^{t+1})\\
 & \times & \PP(U_{1:N}^{t+1}|C_f^{t+1})]]
\end{eqnarray}
or we can sample $U_{1:N}^{t+1}$ from $P(C_f^{t+1})$ or even from $c_f^{t+1}$ (a realization of $C_f^{t=1}$) if we ask $\PP(C_f^{t+1} | eu_{1:M}^t, tt^t, \lambda = 1)$. Note that here we do not know fully neither the value of $ETT$ nor of $C_t$ so we take the most information that we have into account. The evaluation of the question is proportional to $|ETT|\times |EC|^2\times |C|^4 = V'\times K'^2 \times K^4$. But we do not have to sum on the 4 types of $C$ in practice: $\PP(C_m|C_t,C_c)$ is a simple linear combination of vectors and $\PP(\lambda=1|C_f,C_m)$ is a linear filter function, so we just have to sum on $C_c$ and practical complexity is proportional to $V'\times K'^2 \times K$. As we have most often $\approx 10$ Gaussian components in our GMM, $K$ or $K'$ are in the order of 10 (5 to 12 in practice), while $V$ and $V'$ are between 211 and 517 as noted above.

To benchmark our clustering, we use a slightly different model in which $\PP(EC^{t+1})=\PP(EC^t)$ (we want to know which army counters the other in a given battle). The question that we will ask is $\PP(C_c^{t+1}|eu_{1:M}^t)$.

\setlength{\tabcolsep}{5pt}
\begin{table}[ht]
\caption{Summary of the clustering evaluation (for GMM), main metrics}
\begin{center}
\begin{footnotesize}
\begin{tabular}{|c|c|cc|cc|cc|cc|cc|cc|}%cc|}
\hline
forces & scores & \multicolumn{2}{|c|}{PvP} & \multicolumn{2}{|c|}{PvT} & \multicolumn{2}{|c|}{PvZ} & \multicolumn{2}{|c|}{TvT} & \multicolumn{2}{|c|}{TvZ} & \multicolumn{2}{|c|}{ZvZ} \\%& \multicolumn{2}{|c|}{Mean amateurs} \\
disparity & in \% & m & ws & m & ws & m & ws & m & ws & m & ws& m & ws \\
\hline

 %& \multicolumn{2}{|l|}{\#battles learn} & & & & & & \\
%\begin{sideways}\parbox{3mm}{\begin{small}1.5\end{small}}\end{sideways}
% & \multicolumn{2}{|l|}{\#battles test} & & & & & & \\
 %& \multicolumn{2}{|c|}{\#test} & & & & & & \\
 %& \begin{sideways}\parbox{3mm}{\begin{small}w\end{small}}\end{sideways}
& heuristic & \textbf{63} & 63 & 58 & 58 & 58 & 58 & \textbf{65} & \textbf{65} & 70 & 70 & 56 & 56 \\
1.1     & \textbf{just prob.} & 54 & 58 & 68 & \textit{72} & 60 & 61 & 55 & 56 & 69 & 69 & 62 & 63 \\ 
    & prob$\times$heuristic & 61 & \textbf{63} & \textbf{69} & \textbf{72} & \textbf{59} & \textbf{61} & 62 & 64 & \textbf{70} & \textbf{73} & \textbf{66} & \textbf{69} \\
\hline
& heuristic & \textbf{73} & 73 & 66 & 66 & \textbf{69} & \textbf{69} & \textbf{75} & 72 & 72 & \textbf{72} & 70 & 70 \\
1.3     & \textbf{just prob.} & 56 & 57 & 65 & \textit{66} & 54 & 55 & 56 & 57 & 62 & 61 & 63 & 61 \\
    & prob$\times$heuristic & 72 & \textbf{73} & \textbf{70} & \textbf{70} & 66 & 66 & 71 & \textbf{72} & \textbf{72} & 70 & \textbf{75} & \textbf{75} \\
\hline
& heuristic & 75 & 75 & 73 & 73 & \textbf{75} & \textbf{75} & 78 & \textbf{80} & \textbf{76} & 76 & 75 & 75 \\
1.5     & \textbf{just prob.} & 52 & 55 & 61 & 61 & 54 & 54 & 55 & 56 & 61 & \textit{63} & 56 & 60 \\
    & prob$\times$heuristic & \textbf{75} & \textbf{76} & \textbf{74} & \textbf{75} & 72 & 72 & \textbf{78} & 78 & 73 & \textbf{76} & \textbf{77} & \textbf{80} \\
\hline
 %& measure & \multicolumn{2}{|c|}{$d$ for $k=0$} & \multicolumn{2}{|c|}{$k$ for $d=1$} & \multicolumn{2}{|c|}{$k$ for $d=2$} & \multicolumn{2}{|c|}{$k$ for $d=3$} \\
\end{tabular}
\label{results}
\caption{Winner prediction scores (in \%) for 3 main metrics. For the left columns (``m''), we considered only military units. For the right columns (``ws'') we also considered static defense and workers. The ``heuristic'' metric is a baseline heuristic for battle winner prediction for comparison using army values, while ``just prob.'' only considers $P(C_c|EC)$ to predict the winner, and ``prob$\times$heuristic'' balances the heuristic's predictions with $\sum_{C_c,EC}P(C_c|EC)P(EC)$.}

\end{footnotesize}
\end{center}
\end{table}
All the results presented in this section represent the nine match-ups (races combinations) in 1 versus 1 (duel) of StarCraft. We worked with a data-set of 7708 replays of highly skilled human players. For each match-up, we set aside 100 test matches, and use the remaining of the dataset for learning. Performance wise, for the biggest dataset (PvT) the learning part takes around 100 second on a 2.8 Ghz Core 2 Duo CPU (and it is easily serializable) for 2408 games (57 seconds to fit and select the GMM, and 42 seconds to fill the probability tables). %The time to parse all the dataset is far larger (617 seconds with a SSD). %Each inference (question) step takes around $YYY$ second. 
We preferred robustness to precision and thus we did not remove outliers: better scores can easily be achieved by considering only stereotypical armies/battles, or more amateur players (which styles are less pronounced). 

\begin{figure}[htp]
\centerline{\includegraphics[width=0.55\columnwidth]{images/GMM_ISO_Z2.png}}
\label{isoz}
\caption{2 dimensional Isomap projection of a small dataset of battles for Zerg (vs Protoss) with most probable Gaussian mixture model components as labels.}
\end{figure}

%\subsection{Benchmark of the Clustering}
To benchmark our clustering methods (home-made, K-means, GMM), we reduced battle observations to clusters for both parties and used the learned $P(C_c|EC)$ to estimate the outcome of the battle. For that, we used battles with \textit{disparities} (the maximum strength ratio of one army over the other) of 1.1 to 1.5. There is an average of 5 battles per game at a 1.3 disparity threshold (more above). A summary of the main metrics is shown in Table~\ref{results}: we can see that predicting battle outcomes (even with a high disparity) with ``just probabilities'' of $P(C_c|EC)$ (without taking the forces into account) gives relevant results. Note that this is a very high level (abstract) view of a battle, we do not consider tactical positions, nor players' attention, actions, etc. We can somehow view the ``just prob.'' results as the military efficiency improvement we can (at least) expect from having the right army composition. Also, without explicitly labeling clusters, one can apply thresholding to special units (observers, arbiters, science vessels...) to generate more specific clusters: we did not include these results here but they sometimes drastically increase prediction scores.

%%% TODO Isomap / PCA

We also plotted the generated clusters in all kind of way, Figure~\ref{isoz} shows a 2D Isomap projection of the battles on a small dataset. Figure~\ref{parallelplot} simply lays out the concept of \textit{army composition} and expert gamers recognized the clusters clearly identified by the clustering (GMM). Figure~\ref{ecknowingecnext} showcases the dynamics of clusters components: $P(EC^t|EC^{t+1}$, for Zerg (vs Protoss) for $\Delta t$ of 2 minutes. As shown in Figure~\ref{bbq_dataflow}, this model is part of the strategy module of our next AIIDE competition bot.
%Finally, in the absence of ground truth to directly evaluate the whole model, we can only 

\begin{figure}[htp]
\centerline{\includegraphics[width=0.7\columnwidth]{images/PvP_small.png}}
%\centerline{\includegraphics[width=2.1\columnwidth]{PvZ2.png}}
\caption{Parallel plot of a small dataset of Protoss (vs Protoss, i.e. in the PvP match-up) army clusters on most important unit types (for the match-up). Each normalized vertical axis represents the percentage of the units of the given unit type in the army composition (we didn't remove outliers, so most top (tip) vertices represent 100\%), except for the rightmost (framed) which links to the most probable GMM component. Note that several traces can (and do) go through the same edge.}
\label{parallelplot}
\end{figure}



%\subsection{Learned Tables}
\begin{figure}[htp]
\centerline{\includegraphics[width=0.55\columnwidth]{images/Z_EC_knowing_ECnext.png}}
\caption{Dynamics of clusters: $\PP(EC^t|EC^{t+1})$ for Zerg}
\label{ecknowingecnext}
\end{figure}

%%% \begin{figure}[htp]
%%% \centerline{\includegraphics[width=0.92\columnwidth]{images/P_ECnext_knowing_ETT_37.png}}
%%% \caption{log $\PP(EC^{t+1}|ETT=37)$ for Protoss}
%%% \label{ecnextknowingett}
%%% \end{figure}
Here, we focused on asking $\PP(U_{1:N}^{t+1}|eu{1:M}^t, tt^t, \lambda=1)$, and evaluated (in the absence of ground truth for full armies compositions) the two key components that are $\PP(U_{1:N}|C)$ (or $\PP(EC_{1:M}|EC)$) and $\PP(C_c|EC)$. Many other questions can be asked: $\PP(TT^t|eu_{1:M}^t)$ can help us adapt our tech tree development to the opponent's army. If we know the opponent's army composition only partially, we can benefit of knowledge about $ETT$ to know what is possible, but also probable, by asking $P(EC^t|Observations)$.


We presented a probabilistic model inferring the best army composition given what was previously seen (from replays, or previous games), integrating adaptation to the opponent with other constraints (tactics). One of the main advantages of this approach is to be able to deal natively with incomplete information, due to player's intentions, and to the fog of war in RTS. The army composition dimensionality reduction (clustering: unsupervised) can be applied to any game and coupled with other techniques, for instance for situation assessment in case-based planning. The results in battle outcome prediction (from few information) shows its situation assessment potential. Finally, we will use this model in our StarCraft AI competition entry bot as it facilitates adaptive decision-making on the production plan. The main research direction is now to find a way to evaluate the full model integrated to a bot, both for performance of the bot and for the fun of a human player.


%%% \subsection{Results on StarCraft}

%%% \begin{quotation}\textit{
%%% However beautiful the strategy, you should occasionally look at the results.
%%% } Winston Churchill\end{quotation}


\section{Discussion}

\subsection{Possible Uses}
Developing beforehand a RTS game AI that specifically deals with whatever strategies the players will come up is very hard. And even if game developers were willing to patch their AI afterwards, it would require a really modular design and a lot of work to treat each strategy. With our model, the AI can adapt to the evolutions in play by learning its parameters from the replay, and it can dynamically adapt during the games by using the reverse question $\PP(BuildTree | Opening,Time)$, or even $\PP(TechTree | Opening, Time)$ if we use a $TechTree$ variable encoding buildings and technology upgrades. This question would give the distribution over technology trees knowing the opening we want to perform at which time. This would allow for the bot to dynamically choose/change build orders.

We could also use this model in a commentary assistant AI. In the StarCraft and StarCraft 2 communities, there are a lot of \gloss{pro-gamer} tournaments that are commented and we could provide a tool for commentators to estimate the probabilities or different openings or technology paths. As in commented poker matches, where the probabilities of different hands are drawn on screen for the spectators, we could display the probabilities of openings. In such a setup we could use more features as the observers and commentators can see everything that happens (upgrades, units) and we limited ourselves to ``key'' buildings in the work presented in this paper.

\subsection{Improvements}

First, our prediction model can be upgraded to have a higher recognition rate: we could reason about $t+1$ explicitly before computing the distribution over possible openings at $t$ and thus compute the distribution over technology trees at $t+1$. Perhaps it would increase the results of $\PP(Opening|Observations)$, but it almost surely would increase $\PP(BuildTree^{t+1}|Observations)$ which is important for late game predictions. We could also make use of more features as we currently only use at most 20 features (only buildings), and never all at once. Perhaps also that incorporating priors per match-up would lead to better results.

Then, we could feed it with \textit{more} replays during the learning by scrapping more progamers level replays websites. Also, we could learn from replays of bot vs bot matches. For the learning part, the labeling of replays is very important, and our labeling methods can be improved. 
We could explore auto-supervised learning \citep{AutoSuperLearning}. 
Clearly, some match-ups are handled better, either in the replays labeling part and/or in the prediction part. Replays could be labeled by humans and we would do supervised learning then. Or they could be labeled by a combination of rules (as in \citep{weberStrat}) and statistical analysis (as the method presented here). Finally, the replays could be labeled by match-up dependent openings (instead of race dependent openings currently) and could contain either the two parts of the opening %(early and late developments) 
or the game time at which the label is the most relevant, as openings are often bimodal (``fast expand into mutas'', ``corsairs into reaver'', etc.).

Finally, a hard problem is detecting the ``fake'' builds of very highly skilled players. Indeed, some progamers have build orders which purpose are to fool the opponent into thinking that they are performing opening A while they are doing B. %For instance they could ``take early gas'' leading the opponent to think they are going to do tech units, not gather gas 
For instance, they could leadthe opponent to think they are going to \textit{tech} 
and perform an early rush instead. We think that this can be handled by our model by changing $\PP(Opening|LastOpening)$ by $\PP(Opening|LastOpening, LastObservations)$ and adapting the influence of the last prediction with regard to the last observations (i.e., we think we can learn some ``fake'' label on replays).
