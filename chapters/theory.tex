\chapter{Our thesis: Bayesian Modeling of Multi-player Games}

Question:
An efficient and evolutive modeling of a player in a multi-player game.
(How do we do...?)

\section{Transversal problems: summing up problems encountered}
\subsection{Player modeling \& plan recognition}
Strategy adaptation
\subsection{Operational team decision making}
Micro / Tactics (squad AI)
\subsection{Planning under uncertainty}
Macro
\subsection{Learning and adaptability}
Multi-scale

\section{Why?}
\begin{itemize}
\item The bot can't cheat (at least it's not fun!)
\item The bot can't assume optimal play from the opponent when the problem is so large
\item The bot can learn from others games, self past games, self current game
\item The bot can't be in the head of your opponent (meta-)
\end{itemize}

\section{How?}
\begin{itemize}
\item Bayesian programming methodology
\item When in doubt, toss the distribution to your neighbour
\item Exploit gameplay/game rules structure
\item Learn and eat data for breakfast
\item Meta- can be solved by being (globally) stateless and applying the same model as self on the opponent with her sensory inputs
\end{itemize}
