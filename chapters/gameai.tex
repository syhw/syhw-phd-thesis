\chapter{Game AI}
%Babbage, Charles -- . . .every game of skill is susceptible of being played by an automaton.
%Minsky, Marvin -- It is not that the games and mathematical problems are chosen because they are clear and simple; rather it is that they give us, for the smallest initial structures, the greatest complexity, so that one can engage some really formidable situations after a relatively minimal diversion into programming. From Semantic Information Processing, p. 12. Cambridge, MA: MIT Press (1968).
% If you'd like to know, I can tell you that in your universe you move freely in three dimensions that you call space. You move in a straight line in a fourth, which you call time, and stay rooted to one place in a fifth, which is the first fundamental of probability. After that it gets a bit complicated, and there's all sort of stuff going on in dimensions thirteen to twenty-two that you really wouldn't want to know about. All you really need to know for the moment is that the universe is a lot more complicated than you might think, even if you start from a position of thinking it's pretty damn complicated in the first place. I can easily not say words like "damn" if it offends you. H2G2

\begin{verse}\textit{
\\
David: What is the primary goal?\\
Joshua: You should know, Professor. You programmed me.\\
David: Oh, come on. What is the primary goal?\\
Joshua: To win the game.\\
} Wargames (1983)\end{verse}
%\lettrine[image=true, lines=3, findent=3pt, nindent=0pt]{lettrines/O.png}{r}
\lettrine{O}{r}
 is it? ``Game AI'', simultaneously a research topic, an industry standard practice, from a staple to a part of the gameplay. Its uses range from character animation, to behavior modeling and strategic play. In this chapter, we will give our educated guess about the goals of game AI, and review what exists for a broad category of games: single player games, abstract strategy games, partial information and/or stochastic games, computer games. Let us then focus on game-play (from a player point of view) characteristics of theses games so that we can enumerate game AI needs. %%% XXX
%\lettrine[image=true, lines=3, findent=3pt, nindent=0pt]{lettrines/W.png}{hat} is game AI? What are the goals of AI in games? What are its characteristics? Why is game AI an interesting subject for research? 

\section{Goals of Game AI}
%\lettrine[lines=1, lhang=.3]{W}{hat} are the goals of game AI?

\subsection{Win}
During the last decade, the video games industry has see the emergence of ``e-sport''. It is the professionalization of certain competitive games at the higher levels, as in sports, with spectators, leagues, sponsors, fans and broadcasts. A %non-exhaustive 
list of major electronic sports games includes (but is not limited to): StarCraft: Brood War, Counter-Strike, Quake III, Warcraft III, Halo, StarCraft II. The first game to have had progamers was StarCraft: Brood War, in Korea, with top players earning more than top soccer players. Nowadays,
Game AI can be used part of the gameplay as a challenge to the player.

%%%Not solved: humans are the best

\subsection{Fun}
%%%Not solved: humans are the most fun to play with

\subsection{Programming}

\section{Single Player Games}
Single player games are not
\subsection{Reflex/Action games}
Patience, Mario, racing, PacMan
\subsection{Puzzles}
Myst, Tetris, point and clicks...

\section{Abstract Strategy Games}
%%% http://en.wikipedia.org/wiki/Game_complexity
%%% http://en.wikipedia.org/wiki/Solved_board_games
\subsection{Tic-tac-toe}
minimax
\subsection{Checkers}
alpha-beta
\subsection{Chess}
+heuristics
\subsection{Go}
+Monte Carlo

\section{Games with some Uncertainty}
\subsection{Monopoly}
%\subsection{Diplomacy}
%\subsection{Bridge}
\subsection{Battleship}
\subsection{Poker}

%\section{Card Games}
\citep{gunn}

\section{FPS}
\begin{itemize}
\item Problem (gameplay)
\item Quake 3 AI (industry standard w/o squad AI) \citep{waveren-02-artificial}
\item Killzone 2, F.E.A.R \citep{orkinGDC_FEAR} (planning), Crysis 2, BF3: industry standards with squad AI
\item Research:
\begin{itemize}
\item \citep{lehy04}
\item \citep{Laird01} (cognitive architecture)
\item others (\citep{Hladky_anevaluation} ANN, ...)
\item UT Challenge (c.f. CIG)
\end{itemize}
\end{itemize}

\section{(MMO)RPGs}
\begin{itemize}
\item Problem (gameplay)
\item Industry (scripts, behavior trees)
\item Research \citep{Cutumisu09}
\item \citep{SYNNAEVE:MMORPG}
\end{itemize}

\section{RTS}
Expanded later.
\begin{itemize}
\item Problem (gameplay)
\item Industry (Broodwar, AoE, Total A)
\item Research
\end{itemize}

\section{Game Characteristics}
\subsection{Combinatory}
\subsection{Partial information}
%%%\subsection{Multiplayer}
%%%\subsection{PvE}
\subsection{Randomness}
\subsection{Time Constant(s)}
\subsection{Learning Curve}
\subsection{Recap}
\begin{sidewaystable}
\begin{tabular}{|l|ccccc|}
\hline 
Game & Combinatory & Vertical cont. & Horizontal cont. & Partial Info. & Randomness \\
\hline
Checkers & $b\approxeq 10; n\approxeq 70$ & none & none & no & no \\
Chess & $b\approxeq 40; n\approxeq 80$ & none & none & no & no \\
Go & $b\approxeq 300; n\approxeq 150$ & none & some & no & no \\
%Monopoly 
%Battleship
Limit Poker & $b\approxeq 3$\footnote{fold,check,raise} $;n/hour \in [20\dots240]$\footnote{number of decisions taken per hour} & some & few & much & much \\
Time Racing & & & & & \\
(TrackMania) & $b\approxeq 50-1,000$\footnote{$\{X \times Y\}$ sampling$\times$50Hz}$;n/min \approxeq 60$ & full & much & no & no \\
FFPS duel & & & & & \\
(Quake III) & $b\approxeq 200-10,000$\footnote{$\{X \times Y \times Z\}$ sampling$\times$50Hz + firing}$;n/min \approxeq 100$\footnote{60 ``continuous move actions''+ 40 (mean) fire actions per sec} & some & much & some & ($\approxeq$)no \\
%World of Warcraft (duel) & $b\approxeq $ & some & much & some & some \\
RTS & & & & & \\
(StarCraft) & $b\approxeq 200$\footnote{atomic dir/unit $\times$ \# units + constructions + productions}$;n/min=APM\approxeq 300$\footnote{for progamers, counting group actions as only one action}& some & some & much & no\\
\hline
\end{tabular}
\end{sidewaystable}

\section{Player Characteristics}
In all these games, knowledge and learning plays a key role. Humans compensate their lack of (conscious) computational power with pattern matching, abstract thinking and efficient memory structures. 
\subsection{Virtuosity}
Skill
\subsection{Deduction}
\subsection{Induction}
\subsection{Decision-Making}
%%%\subsection{Psychology}
\subsection{Recap}
%%% https://en.wikipedia.org/wiki/Cognition
\begin{sidewaystable}
\begin{tabular}{|l|ccccccc|}
\hline 
Game & Virtuosity & Deduction & Induction & Decision-Making & \multicolumn{3}{c|}{Knowledge} \\
     & (sensory-motor) & (analysis) & (abstraction) & (acting) & game & map & opponent \\
Checkers & 0 & ++ & + & 0 & ++&0&+ \\
Chess & 0 & +++ & + & 0 & ++&0&+ \\
Go & 0 & +++ & ++ & 0 & ++&0&+ \\
Limit Poker & 0 & ++ & ++ & +++ & ++&0&++ \\
Time Racing & & & & & & & \\
(TrackMania) & +++ & 0 & 0 & 0 & +&++&0 \\
FFPS duel & & & & & & & \\ 
(Quake III) & +++ & + & 0 & + & +&++&+ \\
RTS & & & & & & & \\
(StarCraft) & +++ & +++ & +++ & +++ & ++&+&++ \\
\hline
\end{tabular}
\end{sidewaystable}

\section{An interesting problem}
\subsection{Simulated but stochastic}
Human players (ally or foes), and sometimes (most of the time) stochasticity in the rules of the game (fog of war, randomness, etc.).
