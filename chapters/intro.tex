\chapter{Introduction}

\adjustmtc
\chaptertoc

\section{Motivations}

Video games AI research is yielding new approaches to a wide range of problems, for instance in RTS: pathfinding, multiple agents coordination, collaboration, prediction, planning and (multi-scale) reasoning under uncertainty. These problems are particularly interesting in the RTS framework because the solutions have to deal with many objects, imperfect information and micro-actions while running in real-time on desktop hardware.

\begin{itemize}
\item Multi-player video games
Adversarial decision-making. Partial information. Real-time. Massive state spaces.
\item RTS games
Expert human play > AI. Data. Competitions.
\end{itemize}

\section{Contributions}

\begin{itemize}
\item A new approach to game AI with:
\begin{itemize}
\item first-class uncertainty.
\item a tractable hierarchical decomposition of problems.
\end{itemize}
\item Integration of learning in a decision-making model. 
\item An autonomous agent for StarCraft (BroodwarBotQ).
\end{itemize}

Finally, with this thesis, we hope to contribute a guide for industry practitioners who would like to have new tools for solving the ever increasing complexity of game AI, and more generally ``huge state space'' and multi-scale AI.

%\subsection{Decentralization}

%\subsection{Learnings}

%\subsection{Hierarchy}

\section{Reading Map}

À la MacKay (Industry vs Research)


