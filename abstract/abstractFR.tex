Dans la première partie de cette thèse, nous détaillons les solutions actuelles aux problèmes qui se posent lors de la réalisation d'une IA de jeu multi-joueur, en donnant un aperçu des caractéristiques calculatoires et cognitives complexes des principaux types de jeux. En partant de ce constat, nous résumons les catégories transversales de problèmes, et nous introduisons comment elles peuvent être résolues par la modélisation bayésienne. Nous expliquons alors comment construire un programme bayésien en partant de connaissances et d'observations du domaine à travers un exemple simple de jeu de rôle. Dans la deuxième partie de la thèse, nous détaillons l'application de cette approche à l'IA de STR, ainsi que les modèles auxquels nous sommes parvenus. Pour le comportement réactif (micro-management), nous présentons un controleur multi-agent décentralisé et temps réel inspiré de la fusion sensori-motrice. Ensuite, nous accomplissons les adaptation dynamiques de nos stratégies et tactiques à celles de l'adversaire en le modélisant à l'aide de l'apprentissage artificiel (supervisé et non supervisé) depuis des traces de joueurs de haut niveau. Ces modèles probabilistes de joueurs peuvent être utilisés à la fois pour la prédiction des décisions/actions de l'adversaire, mais aussi à nous-même pour la prise de décision si on substitue les entrées par les notres. Enfin, nous expliquons l'architecture de notre joueur robotique de StarCraft, et nous précisions quelques détails techniques d'implémentation.
Au delà des modèles et de leurs implémentations, il y a trois contributions principales: la reconnaissance de plan et la modélisation de l'adversaire par apprentissage artificiel, en tirant partie de la structure du jeu, la prise de décision multi-échelles en présence d'informations incertaines, et l'intégration des modèles bayésiens au contrôle temps réel d'un joueur artificiel.

